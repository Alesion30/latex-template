%#! latexmk -silent -outdir=out temp.tex
\documentclass[a4paper,11pt]{jsarticle}

% 数式
\usepackage{amsmath,amsfonts}
\usepackage{bm}

% 画像
\usepackage[dvipdfmx]{graphicx}

%----------------------------------------------------------------------
% 文書基本情報
%----------------------------------------------------------------------
% タイトル
\title{vscodeで \LaTeX 環境を構築する}

% 著者
\author{UserName}

% 日付
\date{2022年3月4日}

%======================================================================
% テキスト開始
%======================================================================
\begin{document}

% 表紙
\maketitle

%======================================================================
% 本文ここから
%======================================================================
\begin{abstract}
本稿は, macとvscodeで\LaTeX 環境を構築する手順について述べている.
\end{abstract}

\section{\LaTeX をインストール}

''brew install mactex-no-gui --cask''をターミナルで実行する. latexmkのインストールが完了すると, ''latexmk temp.tex''でコンパイルできるようになる.


\end{document}
